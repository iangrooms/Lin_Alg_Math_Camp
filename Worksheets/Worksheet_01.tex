  \documentclass[11pt,fleqn]{article}
 \usepackage{amsmath}
 \usepackage{amssymb}
 \usepackage{bm}
 \usepackage[margin=1in]{geometry}
 \usepackage{multicol}
\usepackage{fancyhdr}
\usepackage{systeme}
\usepackage{enumerate}
\usepackage{nicefrac}

\pagestyle{fancy}
\fancyfoot{}
\lhead{{\bf LinAlg Math Camp}}
\rhead{{\bf Worksheet \#1}}
\chead{}
\cfoot{\thepage}

\newcommand{\mat}[1]{\mathbf{#1}}

% Actual document starts here 
% ======================================================================================
\begin{document}

\begin{enumerate}
\item Suppose that you are given $\mat{A}$, $\mat{B}$, and $\vec{x}$. Explain how to evaluate the expression $(\mathbf{A}+\mathbf{B}^{-1})^{-1}\vec{x}$ without computing the inverse of any matrices.
\item Without using the Wronskian, prove that the following three functions are linearly independent
\[p_1(x) = 1,\;p_2(x) = 1 - x,\;p_3(x) = 1 - x + x^2.\]
\item Consider the vector spaces of polynomials $V$ of degree at most 2 and $W$ of degree at most 3. 
	\begin{itemize}
	\item[(a)] Find the matrix representation of the following linear function from $V$ to $W$
	\[L[p] = 2x\frac{\mathrm{d}p}{\mathrm{d}x}\]
	with the monomial basis for both spaces.
	\item[(b)] Find the matrix representation of the same linear function with respect to the Chebyshev basis for both spaces: $1, x, 2x^2 - 1$, and for $W$ also $4x^3-3x$.
	\end{itemize}
\item Consider the function $L:\mathbb{R}^n\to\mathbb{R}^n$ that maps the initial condition to the final condition at time $t=\tau$ for a linear, non-autonomous, homogeneous system of differential equations:
\[\frac{\mathrm{d}\vec{x}}{\mathrm{d}t} = \mathbf{A}(t)\vec{x},\;\;\vec{x}(0) = \vec{x}_0.\]
	\begin{itemize}
	\item[(a)] Prove that this function is linear.
	\item[(b)] Infer from (a) that $\vec{x}(\tau) = \Pi(\tau)\vec{x}_0$ (no work required). Write down a system of ODEs that $\Pi(\tau)$ must solve (explain why).
	\end{itemize}
\item Let $\vec{0}\neq \vec{v}\in\mathbb{R}^m$ and $\vec{0}\neq\vec{w}\in\mathbb{R}^n$. Prove that the matrix $\vec{v}\vec{w}^T$ has rank one.
\item Find the rank of this $n\times n$ matrix
\[\mathbf{I} - \frac{1}{n}\mathbf{1}\]
where $\mathbf{1}$ is a matrix whose every entry is 1.
\item (Sherman-Morrison) Let $\mathbf{A}$ be an $n\times n$ invertible matrix and let $\vec{u},\vec{v}\in\mathbb{R}^n$. Without even {\it thinking} about using a determinant, prove
	\begin{itemize}
	\item[(a)] that $\mathbf{A}+\vec{u}\vec{v}^T$ is invertible iff $1+\vec{v}^T\mathbf{A}^{-1}\vec{u}\neq0$, and
	\item[(b)] in that case
	\[\left(\mathbf{A}+\vec{u}\vec{v}^T\right)^{-1} = \mat{A}^{-1} - \frac{1}{1+\vec{v}^T\mathbf{A}^{-1}\vec{u}}\mat{A}^{-1}\vec{u}\vec{v}^T\mat{A}^{-1}.\]
	(Hint: Use the expression given in (b) to prove one direction in (a). To prove the other direction, consider the vector $\mathbf{A}^{-1}\vec{u}$.)
	\end{itemize}
\item Prove that if $\vec{v}_1,\ldots,\vec{v}_r$ are a basis for the corange of $\mathbf{A}$, then $\mathbf{A}\vec{v}_1,\ldots,\mathbf{A}\vec{v}_r$ are a basis for the range of $\mathbf{A}$.
\end{enumerate}
 

\end{document}