  \documentclass[11pt,fleqn]{article}
 \usepackage{amsmath}
 \usepackage{amssymb}
 \usepackage{bm}
 \usepackage[margin=1in]{geometry}
 \usepackage{multicol}
\usepackage{fancyhdr}
\usepackage{systeme}
\usepackage{enumerate}
\usepackage{nicefrac}

\pagestyle{fancy}
\fancyfoot{}
\lhead{{\bf LinAlg Math Camp}}
\rhead{{\bf Worksheet \#2}}
\chead{}
\cfoot{\thepage}

\newcommand{\mat}[1]{\mathbf{#1}}

% Actual document starts here 
% ======================================================================================
\begin{document}

\begin{enumerate}
\item Let $\vec{x}$ and $\vec{y}$ be vectors in an inner product space and let $\|\cdot\|$ be derived from the inner product. Prove that $\vec{x}\perp\vec{y}$ if and only if $\|\vec{y}\|\le\|\vec{y}+t\vec{x}\|$ for all scalars $t\in\mathbb{R}$.
\item Prove that $\langle\mathbf{A},\mathbf{B}\rangle=$ Tr$\left(\mathbf{AB}^T\right)$ is an inner product on the space of real $m\times n$ matrices. (Note that this implies that $(\langle\mathbf{A},\mathbf{A}\rangle)^{1/2}$ defines a norm on real $m\times n$ matrices; this norm is called the Frobenius norm. It is not an operator norm. It is not the matrix 2-norm.)
\item Suppose that you have a norm $\|\cdot\|$ on a real vector space but you don't know if it comes from an inner product.
Prove that if the norm does come from an inner product, then the inner product must be
	\[\langle\vec{x},\vec{y}\rangle = \frac{1}{4}\left(\|\vec{x}+\vec{y}\|^2-\|\vec{x}-\vec{y}\|^2\right).\] This is called a polarization identity. If the norm does not come from an inner product, which of the three properties of an inner product will it fail to satisfy?
\item Consider the vector space of polynomials of degree $\le n$ and the inner product $\langle p,q\rangle = \int_0^1 p(x)q(x)\mathrm{d}x$. Find an expression for the Gram matrix associated with the monomial basis for this space. This matrix is called the Hilbert matrix. Explain why it is positive definite.
\item Let $\mathbf{K}$ be symmetric and positive definite and assume it has the factorization $\mathbf{LDL}^T$ from the notes. Prove that the diagonal elements of $\mathbf{D}$ must be positive.
\item Every square matrix $\mathbf{A}$ can be uniquely decomposed into a sum of a symmetric and an antisymmetric matrix:
\[\mathbf{A} = \frac{\mathbf{A}+\mathbf{A}^T}{2} + \frac{\mathbf{A} - \mathbf{A}^T}{2}.\]
Show that if $\mathbf{J}$ is real and antisymmetric, then $\vec{x}^T\mathbf{J}\vec{x} = 0$ for every $\vec{x}\in\mathbb{R}^n$.
\item Riesz representation theorem:
	\begin{itemize}
	\item[(a)] Consider the linear functional $L[x\vec{e}_1+y\vec{e}_2+z\vec{e}_3] = y$ and the inner product defined by
	\[\langle\vec{x},\vec{y}\rangle = \vec{x}^T\mathbf{K}\vec{y}\text{ where }\mathbf{K} = \left[\begin{array}{ccc}2&1&\\1&2&1\\&1&2\end{array}\right].\]
	Find the vector $\vec{z}\in\mathbb{R}^3$ such that $L[\vec{x}] = \langle\vec{x},\vec{z}\rangle$ for all $\vec{x}$. (Use {\tt numpy} or Mathematica if you don't want to solve by hand.)
	\item[(b)] Consider the linear functional $L[p] = \int_0^1 e^{-x}p(x)\mathrm{d}x$ defined on polynomials of degree $\le 2$ and using the inner product $\langle p,q\rangle = \int_0^1p(x)q(x)\mathrm{d}x.$ Find the polynomial $r(x)$ of degree $\le 2$ such that $L[p] = \langle p,r\rangle$ for every polynomial $p$ of degree $\le 2$.  (Use Mathematica or similar to speed up the analysis.)
	\end{itemize}
\item Consider the linear function $L[f](x) = \mathrm{d}[xf'(x)]/\mathrm{d}x$ acting on polynomials of degree $\le 2$, and the inner product $\langle p,q\rangle = \int_0^1 x p(x) q(x)\mathrm{d}x$. What is the adjoint of $L$ with respect to this inner product? Hint: Express your solution as a map from monomial coefficients to monomial coefficients, and use Mathematica or numpy to aid the computation.
\end{enumerate}
 

\end{document}